\documentclass{article}
\usepackage{indentfirst,amsmath, amsthm, amssymb, bm, caption, CJK, graphicx, hyperref, mathrsfs, color}

\begin{document}

    $$
    \phi_1+\phi_2=\phi=\mathrm{const}.\ \ \ \ I=mR^2=\rho R^3\phi.
    $$

    %%若天**力

    $$
    E=\rho R\phi R(1-\cos\varphi)+\frac{1}{2}\rho R^3\phi\dot{\theta}^2
    $$

    $$
    E=\rho R\phi R(1-\cos(\phi_2-\frac{\phi}{2}))+\frac{1}{2}\rho R^3\phi\dot{\phi_2}^2
    $$

    $$
    -\rho\phi Rg^2\sin(\phi_2-\frac{\phi}{2})=\rho R^3\phi\ddot{\phi}_2
    $$

    $$
    \ddot{\phi}_2=-\frac{g}{R}\sin(\phi_2-\frac{\phi}{2})
    $$

    $$
    \mathrm{d}m=\rho R\mathrm{d}\theta 
    $$

    $$
    T\mathrm{d}\theta+N+\mathrm{d}mg\cos\theta=\mathrm{d}m\omega^2r
    $$

    $$
    \mathrm{d}T=\mathrm{d}mg\cdot\sin\theta 
    $$

    $$
    \Rightarrow\mathrm{d}T=\rho Rg\sin\theta\mathrm{d}\theta+\underbrace{\rho gR\sin(\phi_2-\frac{\phi}{2})}_{\mathrm{constant}}\mathrm{d}\theta
    $$

    $$
    T|_{\phi_2}=T|_{\phi_1}=0
    $$

    \begin{equation}
        T(\theta)=\rho gR(\cos{\phi_2}-\cos\theta)+\rho gR\sin(\phi_2-\frac{\phi}{2})(\phi_2-\theta)
    \end{equation}

    \begin{equation}
        a_{*\&*^\&}=\frac{N}{\mathrm{d}m}=\omega^2r-g\cos\theta-T\frac{\mathrm{d}\theta}{\mathrm{d}m}=\omega^2r-g\cos\theta-\frac{T}{\rho R}
    \end{equation}

    \begin{enumerate}

        \item[\textbf{Case 1:}] 
        
        $\theta=0,\phi_1=0,\phi_2=\phi,T\approx 0$.

        \item[\textbf{Case 2:}] 
        
        $\theta=0,\phi_1=\phi,\phi_2=0,T\approx 0$.

        \item[\textbf{Case 3:}] 
        
        $\theta=0,\phi_1=\frac{\phi}{2},\phi_2=\frac{\phi}{2},T\approx \rho gR(\cos\frac{\phi}{2}-1)$.

        Substituting Eq.(2) into the equation above, we have

        $$
        a_{\lambda}=\omega^2_{\mathrm{top}}r-g\cos\frac{\phi}{2}
        $$

        $$
        \Delta \omega^2=\frac{2g}{R}(1-\cos\frac{\phi}{2})
        $$

        $$
        a_{\lambda}=\omega^2r-2g(1-\cos\frac{\phi}{2})-g\cos\frac{\phi}{2}=\omega^2r-2g+g\cos\frac{\phi}{2}<\omega^2r-g
        $$

        $$
        \mathrm{d}m=\rho R\mathrm{d}\theta
        $$

        Total length is $\phi$.

        So the accelerate of the two sides is larger than that of the middle.

    \end{enumerate}

















\end{document}